\documentclass{article}
\usepackage[utf8]{inputenc}
\usepackage[margin=1.0in]{geometry}
\usepackage{amsmath}
\usepackage{amssymb}
\usepackage{fancyhdr}
\usepackage{physics}
\usepackage{wrapfig}
\usepackage{hyperref}
\usepackage{multirow}
\usepackage{amsthm}
\usepackage{pgfplots}

\pgfplotsset{compat=1.16}



\renewcommand{\thesubsection}{\thesection\Alph{subsection}}
\renewcommand\qedsymbol{\square}



\title{Classical Mechanics  PS3}
\author{Joe Crowley}
\date{October 2020}

\pagestyle{fancy}
\renewcommand{\headrulewidth}{0pt}
\renewcommand{\footrulewidth}{1pt}

\fancyhf{}
\rhead{
Joe Crowley \\
Physics 205 \\
Problem Set 3\\
}
\rfoot{Page \thepage}

\begin{document}  

\section{GPS 3.11}
\textit{Two particles move about each other in circular orbits under the influence of gravitational forces, with a period $\tau$. Their motion is suddenly stopped at a given instant of time, and they are then released and allowed to fall into each other. Prove that they collide after a time $\tau / 4 \sqrt{2}$.}


\begin{equation*}
    L=\frac{1}{2} \mu\left(r^{2}+r^{2} \dot{\theta}^{2}\right)+\frac{k}{r}
\end{equation*}
\begin{align*}
    \frac{d}{d t}\left(\frac{\partial L}{\partial \dot{r}}\right)-\frac{\partial L}{\partial r}&=0\\
    \mu \ddot{r}-\mu r \dot{\theta}^{2}+\frac{k}{r^{2}}&=0\\
    &\\
    \frac{d}{d t}\left(\frac{\partial L}{\partial \dot{\theta}}\right)-\frac{\partial L}{\partial \theta}&=0\\
    \frac{\partial L}{\partial \dot{\theta}}&=\operatorname{const}\\
    \operatorname{mr}^{2} \dot{\theta}=\ell
\end{align*}
Suddenly, $\dot{\theta}=0$. 

\begin{align*}
    \ddot{r}&=-\frac{k}{\mu r^{2}}\\
    \ddot{r} \dot{r}&=-\frac{k}{\mu r^{2}} \dot{r}\\
    \frac{1}{2} r^{2}&=\frac{k}{\mu r}+C_{1}\\
    \dot{r}^{2}&=\frac{2 k}{\mu r}+C_1
\end{align*}

Using initial conditions $r = r_0, \dot r = 0$, $C=-\frac{2 K}{\mu r_{0}}$.

\begin{align*}
    \dot{r}^{2}&=\frac{2 k}{\mu}\left(\frac{1}{r}-\frac{1}{r_{0}}\right)\\
    r&=\sqrt{\frac{2 k}{\mu}}\left(\frac{1}{r}-\frac{1}{r_{0}}\right)^{\frac{1}{2}}\\
\end{align*}

\begin{align*}
    \sqrt{\frac{\mu}{2 k}}\left(\frac{1}{r}-\frac{1}{r_{0}}\right)^{-\frac{1}{2}} d r&=dt\\
    -\sqrt{\frac{\mu}{2 k}}\left(r_{0} r \sqrt{\frac{1}{r}-\frac{1}{r_{0}}}-r^{\frac{3}{2}} \tan ^{-1}\left(\sqrt{r_0\left(\frac{1}{r}-\frac{1}{r_{0}}\right)}\right)\right)\Big|_{r_{0}} ^{0}&=\Delta t\\
    \frac{a^{\frac{3}{2}} \pi \sqrt{\frac{\mu}{k}}}{2 \sqrt{2}}&=\Delta t
\end{align*}

Using $\tau=2 \pi a^{\frac{3}{2}} \sqrt{\frac{\mu}{k}}$,

$$\boxed{\Delta t=\frac{\tau}{4 \sqrt{2}}}$$

\newpage



\section{GPS 3.12}
\textit{Suppose that there are long-range interactions between atoms in a gas in the form of central forces derivable from a potential
$$
U(r)=\frac{k}{r^{m}}
$$
where $r$ is the distance between any pair of atoms and $m$ is a positive integer. Assume further that relative to any given atom the other atoms are distributed in space such that their volume density is given by the Boltzmann factor:
$$
\rho(r)=\frac{N}{V} e^{-U(r) / k T}
$$
where $N$ is the total number of atoms in a volume $V$. Find the addition to the virial of Clausius resulting from these forces between pairs of atoms, and compute the resulting correction to Boyle's law. Take $N$ so large that sums may be replaced by integrals. While closed results can be found for any positive $m,$ if desired, the mathematics can be simplified by taking $m=+1$}
\newpage


\section{GPS 3.19}
\textit{A particle moves in a force field described by
$$
F(r)=-\frac{k}{r^{2}} \exp \left(-\frac{r}{a}\right)
$$
where $k$ and $a$ are positive.}


\subsection{}
\textit{Write the equations of motion and reduce them to the equivalent one-dimensional problem. Use the effective potential to discuss the qualitative nature of the orbits for different values of the energy and the angular momentum.}

\begin{align*}
    L &= T - v \\
     &= \frac{1}{2} m\left(\dot{r}^{2}+r^{2} \dot{\theta}^{2}\right)+\int F \cdot d r\\
\end{align*}
Since $L$ has no explicit time dependence, and the coordinates are natural, the hamiltonian is conserved and is the total energy. 

\begin{align*}
    E&=\frac{1}{2} m\left(r^{2}+r^{2} \dot{\theta}^{2}\right)-\int F \cdot d r\\
     &\frac{d}{d t} E=m\left(\ddot{r} \ddot{r}+r \dot{r} \dot{\theta}^{2}+r^{2} \dot{\theta} \ddot{\theta}\right)-F \dot{r}\\
\end{align*}

From the Euler-Lagrange equation for $\theta$, 

\begin{align*}
    \dot \theta &= \frac{l}{m r^2}\\
    \ddot \theta &= -\frac{2l}{m r^3}\dot r\\
\end{align*}

\begin{align*}
    \dot{r} \ddot{r}&=\frac{F}{m} \dot{r}-r \dot{r} \frac{l^{2}}{m^{2} r^{4}}-r^2 \frac{l}{m r^{2}}\left(-\frac{2 l}{m r^{3}}\right) \dot{r}\\
    
\end{align*}

\subsection{}
\textit{Show that if the orbit is nearly circular, the apsides will advance approximately by $\pi \rho / a$ per revolution, where $\rho$ is the radius of the circular orbit.}



\newpage


\section{GPS 3.21}
\textit{Show that the motion of a particle in the potential field
$$
V(r)=-\frac{k}{r}+\frac{h}{r^{2}}
$$
is the same as that of the motion under the Kepler potential alone when expressed in terms of a coordinate system rotating or precessing around the center of force.

For negative total energy, show that if the additional potential term is very small compared to the Kepler potential, then the angular speed of precession of the elliptical orbit is
$$
\dot{\Omega}=\frac{2 \pi m h}{l^{2} \tau}
$$
The perihelion of Mercury is observed to precess (after correction for known planetary perturbations) at the rate of about $40^{\prime \prime}$ of arc per century. Show that this precession could be accounted for classically if the dimensionless quantity
$$
\eta=\frac{h}{k a}
$$
(which is a measure of the perturbing inverse-square potential relative to the gravitational potential) were as small as $7 \times 10^{-8}$. (The eccentricity of Mercury's orbit is $0.206,$ and its period is 0.24 year.)}

\newpage
\section{GPS 3.22}
\texit{The additional term in the potential behaving as $r^{-2}$ in Exercise 21 looks very much like the centrifugal barrier term in the equivalent one-dimensional potential. Why is it then that the additional force term causes a precession of the orbit, while an addition to the barrier, through a change in $l,$ does not?}

\end{document}
